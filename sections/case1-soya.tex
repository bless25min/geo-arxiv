\section{Case Study: GEO Implementation on SOYA Course Review Content}

To validate the efficacy of GEO under production-like conditions, we applied a three-phase optimization to the page titled ``SOYA \textbf{\CJKfamily{bsmi}課程評價}” hosted on \texttt{massenlighten.com}. Implementation details are documented in full at \cite{massenlighten2024blog}.

\subsection{Pre-Optimization Baseline}
Before intervention, ChatGPT and Perplexity were queried in incognito visitor mode with the term ``SOYA 課程評價.”
\begin{itemize}
  \item \textbf{ChatGPT:} 0 out of 8--10 segments cited the page
  \item \textbf{Perplexity:} 0 out of 6 content blocks referenced the source
\end{itemize}
Despite first-page ranking in traditional search, generative engines failed to cite the site---suggesting structural citation failure.

\subsection{Optimization Process}
Changes were deployed across three weeks:
\begin{itemize}
  \item \textbf{Week 1:} Semantic anchoring (title rewriting, summary paragraph, intro block)
  \item \textbf{Week 2:} Context expansion (paraphrases, synonyms, term layering)
  \item \textbf{Week 3:} Recomposition (FAQ, modularization, citation-ready definitions)
\end{itemize}
Each change aligned with one GEO semantic layer.

\subsection{Post-GEO Evaluation}
Querying under the same visitor settings yielded:
\begin{center}
\begin{tabular}{|l|c|c|}
\hline
\textbf{System} & \textbf{Segments Returned} & \textbf{Segments Citing SOYA Page} \\
\hline
ChatGPT Run 1 & 9 & 7 \\
ChatGPT Run 2 & 8 & 7 \\
Perplexity Run 1 & 6 & 5 \\
Perplexity Run 2 & 6 & 5 \\
\hline
\end{tabular}
\end{center}

Average citation rate across 4 sessions: \textbf{77.1\%}. Segments cited included mid-body paragraphs and FAQ entries.

\subsection{Implications}
\begin{itemize}
  \item Modular restructuring substantially improved visibility
  \item Schema and semantic clarity allowed correct topical indexing
  \item Citation occurred despite no backlink gain or metadata tuning
\end{itemize}
These outcomes align with GEO-Bench results \cite{aggarwal2024geo}, supporting the claim that semantic structure---not SEO rank---determines generative citation likelihood.
