\section{Introduction}

In recent years, the landscape of online search and information retrieval has undergone a structural transformation. The rise of large language models (LLMs) such as GPT-4, Claude, and Gemini has catalyzed the development of generative search engines (GEs)—systems that directly synthesize answers from multiple sources rather than providing ranked hyperlinks. As a result, conventional Search Engine Optimization (SEO) practices, which focus on ranking positions and click-through rates, are becoming increasingly insufficient for ensuring content visibility.

According to data from BrightEdge’s 2024 AI Search Impact Report, the introduction of Google SGE (Search Generative Experience) led to a 49\% increase in visibility for AI-generated summaries, while the total click-through rate for organic search results dropped by 30\% \cite{lodolce2024gartner}. Concurrently, Search Engine Land reported that nearly 60\% of queries now result in “zero-click searches,” where users consume the AI-generated response without clicking any link \cite{lodolce2024gartner}. In this new paradigm, visibility is no longer about being ranked first—it is about being cited.

This paradigm shift has profound implications for content creators, researchers, and digital marketers. In the era of generative engines such as ChatGPT Search, Perplexity AI, and Google SGE, content must be not only discoverable but also understandable and quotable by AI systems. This requirement demands a rethinking of optimization techniques—a new discipline known as \textbf{Generative Engine Optimization (GEO)}.

First introduced by Aggarwal et al. \cite{aggarwal2024geo}, GEO is a framework that defines visibility not in terms of ranking, but in terms of a source’s influence within an AI-generated response. Their research introduced GEO-Bench, a benchmark dataset containing 10,000 queries across diverse domains, and demonstrated that targeted GEO strategies—such as quotation insertion, statistical evidence, and fluency enhancement—can improve citation likelihood by up to 40\% in generative engines like Perplexity and GPT-3.5-turbo.

However, existing research has primarily focused on general-purpose web content or e-commerce listings. There remains a lack of domain-specific experimentation, practical implementation guidelines, and field-tested frameworks for applying GEO in the real world. More critically, the boundary between GEO and SEO remains poorly defined, particularly in situations where improved citation may result indirectly from improved ranking.

In this paper, we extend the foundational work of GEO by introducing a reproducible, theory-grounded framework that systematically addresses how content can be optimized for citation by generative engines. Our contributions are threefold:
\begin{itemize}
  \item We propose a three-layer semantic visibility model—comprising Semantic Anchoring, Context Triggering, and Pragmatic Recomposition—that explains how content is parsed and reused by LLMs.
  \item We translate this model into an actionable implementation framework, combining HTML structure, Schema.org markup, modular design, and internal semantic linking strategies.
  \item We present two real-world experiments: one focused on improving the visibility of a commercial course review page (SOYA), and another designed to isolate SEO effects by introducing a low-profile query term (``廖天佑'') with no prior web footprint. The results show measurable increases in generative citation despite limited Google ranking, supporting the validity of GEO’s core principles.
\end{itemize}

This paper is structured as follows. Section 2 surveys related work on GEO, SEO, and LLM citation mechanisms. Section 3 presents the theoretical foundation for our semantic visibility model. Section 4 details our methodology and implementation pipeline. Section 5 and 6 present empirical results from the two case studies. Section 7 discusses limitations and practical implications. Section 8 concludes with future directions, including automated GEO scoring tools and reinforcement-based visibility optimization.
