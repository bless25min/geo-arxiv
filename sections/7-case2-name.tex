\section{Control Experiment: Name-Based GEO Visibility Test}

To isolate the effects of Generative Engine Optimization (GEO) from traditional Search Engine Optimization (SEO), we conducted a single-subject case study using the author’s name, “廖天佑,” as the query term. Unlike product or course-related keywords, personal names are less susceptible to keyword targeting or backlink manipulation, making them a suitable context for testing semantic-layer interventions under low-SEO influence.

\subsection{Motivation and Hypothesis}

Prior to this experiment, searching “廖天佑” on ChatGPT and Perplexity produced zero references to the author. Instead, both engines returned results about a different individual---a well-known student with extensive news coverage. This high-ambiguity, low-authority setting provided an ideal stress test for GEO strategies.

We hypothesized that even in the presence of a dominant competing entity and low page rank, semantic anchoring and modular structuring alone could elevate visibility in generative responses.

\subsection{Implementation Pipeline (7/6--7/15)}

The author executed a full-stack GEO deployment using the following sequence:

\begin{enumerate}
  \item Used ChatGPT voice transcription to capture biography and work history.
  \item Structured personal notes and knowledge via ChatGPT linker and research tools.
  \item Exported content as static HTML using Claude 3’s Project Knowledge tool.
  \item Published content on GitHub Pages with semantic interlinking (Pillar/Cluster/Mini nodes).
  \item Embedded \texttt{Person}, \texttt{FAQPage}, and \texttt{Article} schema using JSON-LD.
  \item Submitted site to Google Search Console on July 6; monitored indexation and ranking.
\end{enumerate}

No SEO enhancements (e.g., backlinks, keyword stuffing, metadata) were used. This isolated the effect of semantic and structural design.

\subsection{Evaluation Results (7/15)}

\begin{itemize}
  \item \textbf{Google Search Console}: Indexed with average position = 10.7.
  \item \textbf{Google organic search:} Author site appeared on page 3 (approx. rank 25--30).
  \item \textbf{ChatGPT:} Returned 9 content segments, 5 of which cited the author’s pages.
  \item \textbf{Perplexity:} Returned 5 content blocks, 3 of which cited the author.
\end{itemize}

Notably, ChatGPT’s response explicitly included:

\textit{“你查詢的 ‘廖天佑’ 在網路上主要有以下兩位代表人物,”}

---referring to both the widely known student and the author, indicating successful semantic disambiguation via Schema and structured content.

\subsection{Interpretation}

This follow-up experiment confirms that:

\begin{itemize}
  \item GEO-structured content achieved over \textbf{55\% citation} without SEO ranking.
  \item Semantic anchoring using \texttt{Person} schema enabled LLM disambiguation.
  \item Modular paragraphing and clear topical labels enabled partial citation even in the presence of a dominant name collision.
\end{itemize}

The results support findings from GEO-Bench \cite{aggarwal2024geo}, indicating that citation in generative systems is governed primarily by semantic extractability and structural clarity rather than by popularity, backlink profile, or keyword dominance.
