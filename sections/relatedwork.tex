\section{Related Work}

The emergence of Generative Engine Optimization (GEO) as a research area is rooted in the intersection of three previously distinct domains: search engine optimization (SEO), large language model (LLM) interpretability, and citation behavior in generative responses.

\subsection{The Limitations of Traditional SEO}

Traditional SEO strategies focused on improving page ranking through on-page keyword optimization and off-page link authority \cite{ziakis2024ai}. However, with the rise of zero-click AI-generated results, SEO signals no longer guarantee visibility \cite{lodolce2024gartner}. Generative engines synthesize information across sources, often omitting direct links unless the source exhibits clear quotability and authority.

\subsection{Citation Behavior in Generative Systems}

Studies by Liu et al. \cite{liu2023verifiability} and Menick et al. \cite{menick2022quotes} emphasize the importance of content clarity and factual support in LLM outputs. They suggest that modular, fluent, and specific content improves citation likelihood. However, these works focus on response generation rather than visibility of specific sources.

\subsection{The GEO Framework}

Aggarwal et al. \cite{aggarwal2024geo} introduce GEO as a black-box optimization paradigm for boosting citation presence in LLM answers. Using their GEO-Bench dataset, they test multiple strategies---including statistics insertion, quotation addition, and authoritative tone---and find visibility improvements up to 40\%. They also propose impression metrics such as Position-Adjusted Word Count.

L\"uttgenau et al. \cite{luttgenau2025beyondseo} extend this by applying GEO methods in the tourism domain via fine-tuned BART models. Their results demonstrate a 30.96\% increase in position-weighted visibility.

\subsection{Gap in Implementation Research}

While benchmarks and citation metrics have been proposed, prior work lacks reproducible field implementations or guidance for content creators. Our paper fills this gap by operationalizing GEO in two distinct scenarios and assessing their empirical impact using real-world generative systems.
