\section{Conclusion and Future Work}

This study formalizes and operationalizes Generative Engine Optimization (GEO) as a reproducible content strategy for visibility in LLM-powered search. Through two real-world deployments---SOYA course content and a name-based disambiguation scenario---we demonstrated measurable citation improvements using only structural and semantic techniques, independent of SEO rank or backlinks.

\textbf{Key contributions:}
\begin{itemize}
  \item A three-layer semantic visibility model integrating anchoring, triggering, and recomposition.
  \item Implementation guidelines using ChatGPT, Claude, GitHub Pages, and Schema.org.
  \item Five evaluation metrics for measuring semantic and structural citation readiness.
  \item Empirical results confirming citation uplift of up to 77.1\% in ChatGPT and Perplexity.
\end{itemize}

These findings affirm that GEO enables visibility parity even in low-authority scenarios and provide a practical foundation for LLM-era content design.

\subsection{Future Directions}
\begin{itemize}
  \item \textbf{Citation Monitoring Tools:} Browser extensions or trackers that detect AI citation inclusion in real-time.
  \item \textbf{Automatic Scoring:} LLM-powered tools for computing AIO, CPS, and SRS automatically from raw HTML.
  \item \textbf{Cross-engine Benchmarking:} Extending this method to other LLM engines (e.g., Claude, Meta AI).
  \item \textbf{RL-based GEO:} Reinforcement learning using citation likelihood as reward signal.
\end{itemize}

As generative engines grow in influence, understanding and influencing how content is ingested and cited becomes a critical capability. GEO offers both a theoretical framework and operational toolkit for this new paradigm of search and citation.
