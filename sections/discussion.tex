\section{Discussion}

The experimental results across both case studies offer empirical validation of the core hypothesis of GEO: that visibility in generative engine responses is primarily governed by content structure, not SEO ranking.

\subsection{From Rank to Reference}

The SOYA experiment showed that a high-ranking page (1st on Google) received \textbf{0 citations} prior to GEO optimization. Conversely, after applying semantic structuring, the same content was cited in \textbf{77.1\%} of answer segments. This shift supports prior claims by Aggarwal et al. \cite{aggarwal2024geo} that citation is better predicted by structured quotability than keyword density.

In the name-based test, a low-ranking site with no prior authority or backlinks achieved over \textbf{55\% citation inclusion}. The success was attributable to name disambiguation via Schema.org \texttt{Person} tags and modular semantic layout.

\subsection{Layer Synergy and Failure Cases}

Experimental observations highlight the synergistic nature of GEO's three layers:
\begin{itemize}
  \item \textbf{Layer 1} ensured correct classification and topical anchoring
  \item \textbf{Layer 2} enabled flexible query matching
  \item \textbf{Layer 3} made content quotable in standalone chunks
\end{itemize}
Weakness in any one layer led to failure modes: good modular design without anchoring caused off-topic retrieval, while strong title structure without modularity resulted in omission from final response.

\subsection{Citation vs. SEO Authority}

SEO methods prioritize rank signals like domain authority and link count. Our findings suggest generative engines de-emphasize these signals in favor of content design. This democratizes visibility: creators with no SEO capital can compete if they follow GEO structuring principles.

\subsection{Limitations}

\begin{itemize}
  \item Generative engines evolve frequently, and these results may change
  \item Citation may be stochastic; results may vary with prompt or time
  \item True ingestion timing of new content by LLMs is not observable
\end{itemize}
These constraints are shared by prior GEO research and consistent with its formulation as a black-box optimization task \cite{aggarwal2024geo}.
