\section{Case Study: GEO Implementation on “SOYA Course Reviews” Content}

To validate the practical efficacy of Generative Engine Optimization (GEO), we conducted a three-week structured implementation on a public-facing web page discussing the “SOYA Course Reviews” . The process was executed in July 2024 and documented step-by-step at \url{https://www.massenlighten.com/post/gpt-%E5%BC%95%E7%94%A8%E5%AF%A6%E4%BD%9C%E5%85%A8%E7%B4%80%E9%8C%84%EF%BC%9A%E4%B8%89%E9%80%B1%E5%BE%9E%E9%9B%9C%E5%87%BA%E7%99%BC%EF%BC%8C%E6%89%93%E9%80%A0-ai-%E8%83%BD%E7%9C%8B%E8%A6%8B%E7%9A%84%E5%85%A7%E5%AE%B9}.

\subsection{Pre-Optimization Baseline}

Prior to optimization, we queried “SOYA Course Reviews” (originally in Chinese) in both \textbf{ChatGPT Search} and \textbf{Perplexity AI} in incognito and non-authenticated visitor mode.

\begin{itemize}
  \item \textbf{ChatGPT:} 0 of 8--10 answer segments cited massenlighten.com.
  \item \textbf{Perplexity:} 0 of 6 response segments cited our content.
\end{itemize}

Although the SOYA page ranked highly on Google (position 1), it was not referenced by generative engines—demonstrating that SEO visibility did not translate into LLM citation.

\subsection{Three-Week GEO Optimization}

We applied structural interventions mapped to each GEO semantic layer:

\begin{itemize}
  \item \textbf{Week 1 (Semantic Anchoring):} Rewrote title and introduction; added summary section using \texttt{<section class="summary">}.
  \item \textbf{Week 2 (Context Triggering):} Inserted paraphrases (e.g., “Course evaluation” vs. “student feedback”), related terminology, and domain synonyms.
  \item \textbf{Week 3 (Pragmatic Recomposition):} Broke paragraphs into 3--5 sentence units, added FAQ blocks and bolded definitions.
\end{itemize}

\subsection{Post-GEO Evaluation Results}

We re-ran the same queries under identical visitor-mode conditions.

\begin{table}[h]
\centering
\begin{tabular}{|l|c|c|}
\hline
\textbf{Engine} & \textbf{Answer Segments} & \textbf{Citing Our Page} \\\hline
ChatGPT (Run 1) & 9 & 7 \\\hline
ChatGPT (Run 2) & 8 & 7 \\\hline
Perplexity (Run 1) & 6 & 5 \\\hline
Perplexity (Run 2) & 6 & 5 \\\hline
\end{tabular}
\caption{Citation results across 4 zero-login visitor sessions (July 2024)}
\end{table}

Average citation ratio across both engines and both runs was \textbf{77.1\%}.

\subsection{Interpretation}

The experiment confirms that:

\begin{itemize}
  \item \textbf{Semantic modularity} was key to citation inclusion;
  \item \textbf{Citation occurred despite no change in Google SEO rank or backlinks};
  \item \textbf{LLMs reused both rewritten paragraphs and FAQ blocks}, showing effective recomposition.
\end{itemize}

These findings match GEO-Bench and BART fine-tuning results reported by Aggarwal et al. \cite{aggarwal2024geo} and L\"uttgenau et al. \cite{luttgenau2025beyondseo}, where modular content with semantic clarity increased AI citation performance even without traditional SEO indicators.
