\section{Conclusion and Future Work}

This study formalizes and operationalizes Generative Engine Optimization (GEO) as a reproducible content strategy for visibility in LLM-powered search. Through two real-world deployments---SOYA course content and a name-based disambiguation scenario---we demonstrated measurable citation improvements using only structural and semantic techniques, independent of SEO rank or backlinks.

\textbf{Key contributions:}
\begin{itemize}
  \item A three-layer semantic visibility model integrating anchoring, triggering, and recomposition.
  \item Implementation guidelines using ChatGPT, Claude, GitHub Pages, and Schema.org.
  \item Five evaluation metrics for measuring semantic and structural citation readiness.
  \item Empirical results confirming citation uplift of up to 77.1\% in ChatGPT and Perplexity.
\end{itemize}

These findings affirm that GEO enables visibility parity even in low-authority scenarios and provide a practical foundation for LLM-era content design.

\subsection{Future Directions}

Several avenues merit further investigation:

\begin{itemize}
  \item \textbf{Citation Monitoring Tools:} Browser extensions or LLM-integrated dashboards that detect if and how a URL is cited across sessions.
  \item \textbf{Automatic GEO Scoring:} Tools that use LLMs (e.g., GPT-4, Claude 3) to compute scores like AIO, CPS, and SRS directly from raw HTML inputs.
  \item \textbf{Cross-Engine Testing:} Replication of these experiments across other generative platforms (e.g., You.com, Brave AI, Meta AI) to test framework generalizability.
  \item \textbf{RL-based Optimization:} Fine-tuning small models on citation-based reward functions to iteratively restructure pages for maximum generative visibility \cite{luttgenau2025beyondseo}.
\end{itemize}

As generative engines grow in influence, understanding and influencing how content is ingested and cited becomes a critical capability. GEO offers both a theoretical framework and operational toolkit for this new paradigm of search and citation.

## 🔁 Postscript: Reverse Experiment – SEO Without GEO

To further validate the necessity of semantic structuring in GEO, we conducted an additional reverse experiment using a low-competition English name: **Tsai Hao Jui**.

We authored a web article on AI citation optimization using this name, deliberately **omitting** the GEO structuring framework (i.e., no modular paragraphs, no Schema.org, no semantic mesh). Due to the rarity of this name, the page quickly ranked first in SEO for "Tsai Hao Jui".

However, the AI search visibility results tell a different story:

- 🔍 **Google**: ranked #1 on name search ✅
- 🤖 **ChatGPT**: returned no information related to the authored article ❌
- 📘 **Perplexity**: cited only the LinkedIn profile, not the article ❌

These results underscore the core distinction:  
**SEO alone may secure ranking, but GEO is essential for citation.**

### 📎 Screenshots:

- `figures/case3-tsaihaojui-chatgpt.png`
- `figures/case3-tsaihaojui-google.png`
- `figures/case3-tsaihaojui-perplexity.png`

This confirms that content not structured semantically, even when ranked highly, is unlikely to be cited or extracted by generative engines.
