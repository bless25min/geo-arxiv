\section{Conclusion}

This study presents and empirically validates a structured framework for Generative Engine Optimization (GEO), introducing a three-layer model of semantic visibility: Semantic Anchoring (Layer 1), Context Triggering (Layer 2), and Pragmatic Recomposition (Layer 3). Through multi-stage interventions and simulations across three real-world scenarios—including a course review site, a name disambiguation case, and a large-scale cross-domain semantic variation test—we observed measurable gains in citation inclusion within AI-generated outputs. Our approach demonstrates that citation is not merely a byproduct of search engine ranking, but a function of how content is semantically structured and expressed.


\section{Conclusion and Future Work}

This study formalizes and operationalizes Generative Engine Optimization (GEO) as a reproducible content strategy for visibility in LLM-powered search. Through two real-world deployments---SOYA course content and a name-based disambiguation scenario---we demonstrated measurable citation improvements using only structural and semantic techniques, independent of SEO rank or backlinks.

\textbf{Key contributions:}
\begin{itemize}
  \item A three-layer semantic visibility model integrating anchoring, triggering, and recomposition.
  \item Implementation guidelines using ChatGPT, Claude, GitHub Pages, and Schema.org.
  \item Five evaluation metrics for measuring semantic and structural citation readiness.
  \item Empirical results confirming citation uplift of up to 77.1\% in ChatGPT and Perplexity.
\end{itemize}

These findings affirm that GEO enables visibility parity even in low-authority scenarios and provide a practical foundation for LLM-era content design.

\subsection{Future Directions}

Several avenues merit further investigation:

\begin{itemize}
  \item \textbf{Citation Monitoring Tools:} Browser extensions or LLM-integrated dashboards that detect if and how a URL is cited across sessions.
  \item \textbf{Automatic GEO Scoring:} Tools that use LLMs (e.g., GPT-4, Claude 3) to compute scores like AIO, CPS, and SRS directly from raw HTML inputs.
  \item \textbf{Cross-Engine Testing:} Replication of these experiments across other generative platforms (e.g., You.com, Brave AI, Meta AI) to test framework generalizability.
  \item \textbf{RL-based Optimization:} Fine-tuning small models on citation-based reward functions to iteratively restructure pages for maximum generative visibility \cite{luttgenau2025beyondseo}.
\end{itemize}

As generative engines grow in influence, understanding and influencing how content is ingested and cited becomes a critical capability. GEO offers both a theoretical framework and operational toolkit for this new paradigm of search and citation.

\section*{Postscript: Reverse Experiment – SEO Without GEO}

To further validate the necessity of semantic structuring in GEO, we conducted a reverse experiment using a low-competition English name: \textbf{Tsai Hao Jui}.

An article on AI citation optimization was authored under this name, but deliberately \emph{excluded} all structural techniques introduced in the GEO framework—no modular paragraphs, no Schema.org markup, and no semantic mesh linking.

Due to the rarity of the name, the page easily achieved the top SEO position for the query ``Tsai Hao Jui''. However, the citation visibility results revealed a clear gap:

\begin{itemize}
  \item \textbf{Google:} The article ranked first for ``Tsai Hao Jui''.
  \item \textbf{ChatGPT:} Returned no information about the authored content.
  \item \textbf{Perplexity:} Cited only a LinkedIn profile; the article content was not referenced.
\end{itemize}

These results confirm that \textbf{SEO ranking alone is insufficient for generative engine citation}. Without semantic extractability and modular design, even high-ranking content remains invisible to LLM-based systems.

Screenshot evidence for this experiment is included as:

\begin{itemize}
  \item \texttt{figures/case3-tsai-chatgpt.png}
  \item \texttt{figures/case3-tsai-google.png}
  \item \texttt{figures/case3-tsai-perplexity.png}
\end{itemize}

This reinforces the core hypothesis of this paper: visibility in generative systems is governed not by surface ranking, but by structured semantic readiness.
