\documentclass[11pt]{article}
\usepackage[margin=1in]{geometry}
\usepackage{times}
\usepackage{graphicx}
\usepackage{hyperref}
\usepackage{amsmath}
\usepackage{natbib}
\usepackage{url}
\usepackage{titlesec}
\titleformat{\section}{\normalfont\Large\bfseries}{\thesection.}{0.5em}{}

\title{GEO Fundamentals: Optimization Strategies for Web Content in the Age of AI Search}
\author{Tianyou Liao \\ Independent Researcher \\ \texttt{liaotianyou.research@gmail.com}}
\date{July 2025}

\begin{document}
\maketitle

\begin{abstract}
This paper introduces a comprehensive framework for Generative Engine Optimization (GEO), a methodology designed to increase the visibility and credibility of web content within AI-powered search engines such as ChatGPT, Perplexity, and Google SGE. In contrast to traditional Search Engine Optimization (SEO), which emphasizes keyword ranking and link building, GEO focuses on content citation frequency, semantic clarity, and structured data readiness to enhance inclusion in generative responses. Based on validated industry benchmarks and academic findings, we articulate a three-layer semantic visibility model, practical implementation architecture, and a set of quantitative indicators to guide GEO deployment. We also propose a hybrid SEO-GEO integration strategy for maximizing cross-channel discoverability and brand authority in the evolving search ecosystem.
\end{abstract}

\section{Introduction}
1. Introduction
In recent years, the landscape of online search and information retrieval has undergone a structural transformation. The rise of large language models (LLMs) such as GPT-4, Claude, and Gemini has catalyzed the development of generative search engines (GEs)—systems that directly synthesize answers from multiple sources rather than providing ranked hyperlinks. As a result, conventional Search Engine Optimization (SEO) practices, which focus on ranking positions and click-through rates, are becoming increasingly insufficient for ensuring content visibility.

According to data from BrightEdge’s 2024 AI Search Impact Report, the introduction of Google SGE (Search Generative Experience) led to a 49% increase in visibility for AI-generated summaries, while the total click-through rate for organic search results dropped by 30%GEO基礎原理:AI搜尋時代的內容優化策略. Concurrently, Search Engine Land reported that nearly 60% of queries now result in “zero-click searches,” where users consume the AI-generated response without clicking any linkGEO基礎原理:AI搜尋時代的內容優化策略. In this new paradigm, visibility is no longer about being ranked first—it is about being cited.

This paradigm shift has profound implications for content creators, researchers, and digital marketers. In the era of generative engines such as ChatGPT Search, Perplexity AI, and Google SGE, content must be not only discoverable but also understandable and quotable by AI systems. This requirement demands a rethinking of optimization techniques—a new discipline known as Generative Engine Optimization (GEO).

First introduced by Aggarwal et al. (2024), GEO is a framework that defines visibility not in terms of ranking, but in terms of a source’s influence within an AI-generated response. Their research introduced GEO-Bench, a benchmark dataset containing 10,000 queries across diverse domains, and demonstrated that targeted GEO strategies—such as quotation insertion, statistical evidence, and fluency enhancement—can improve citation likelihood by up to 40% in generative engines like Perplexity and GPT-3.5-turbo2024年ARXIV的GEO研究.

However, existing research has primarily focused on general-purpose web content or e-commerce listings. There remains a lack of domain-specific experimentation, practical implementation guidelines, and field-tested frameworks for applying GEO in the real world. More critically, the boundary between GEO and SEO remains poorly defined, particularly in situations where improved citation may result indirectly from improved ranking.

In this paper, we extend the foundational work of GEO by introducing a reproducible, theory-grounded framework that systematically addresses how content can be optimized for citation by generative engines. Our contributions are threefold:

We propose a three-layer semantic visibility model—comprising Semantic Anchoring, Context Triggering, and Pragmatic Recomposition—that explains how content is parsed and reused by LLMs.

We translate this model into an actionable implementation framework, combining HTML structure, Schema.org markup, modular design, and internal semantic linking strategies.

We present two real-world experiments: one focused on improving the visibility of a commercial course review page (SOYA), and another designed to isolate SEO effects by introducing a low-profile query term (“廖天佑”) with no prior web footprint. The results show measurable increases in generative citation despite limited Google ranking, supporting the validity of GEO’s core principles.

This paper is structured as follows. Section 2 surveys related work on GEO, SEO, and LLM citation mechanisms. Section 3 presents the theoretical foundation for our semantic visibility model. Section 4 details our methodology and implementation pipeline. Section 5 and 6 present empirical results from the two case studies. Section 7 discusses limitations and practical implications. Section 8 concludes with future directions, including automated GEO scoring tools and reinforcement-based visibility optimization.
\begin{itemize}
  \item We propose a three-layer semantic visibility model---comprising Semantic Anchoring, Context Triggering, and Pragmatic Recomposition---that explains how content is parsed and reused by LLMs.
  \item We translate this model into an actionable implementation framework, combining HTML structure, Schema.org markup, modular design, and internal semantic linking strategies.
  \item We present two real-world experiments: one focused on improving the visibility of a commercial course review page (SOYA), and another designed to isolate SEO effects by introducing a low-profile query term (``\textit{\textbf{\CJKfamily{bsmi}廖天佑}}'') with no prior web footprint. The results show measurable increases in generative citation despite limited Google ranking, supporting the validity of GEO's core principles.
\end{itemize}

This paper is structured as follows. Section 2 surveys related work on GEO, SEO, and LLM citation mechanisms. Section 3 presents the theoretical foundation for our semantic visibility model. Section 4 details our methodology and implementation pipeline. Sections 5 and 6 present empirical results from the two case studies. Section 7 discusses limitations and practical implications. Section 8 concludes with future directions, including automated GEO scoring tools and reinforcement-based visibility optimization.

\section{Related Work}

The emergence of Generative Engine Optimization (GEO) as a research area is rooted in the intersection of three previously distinct domains: search engine optimization (SEO), large language model (LLM) interpretability, and citation behavior in generative responses.

\subsection{Limitations of Traditional SEO}

Traditional SEO practices—such as keyword density control, link building, and HTML tag optimization—have long focused on improving a page’s visibility in ranked search results. Search engine optimization (SEO) strategies are typically divided into on-page and off-page methods, aimed at improving keyword rankings, link authority, and click-through rates.
However, the advent of zero-click search has undermined the assumptions underpinning these methods.

BrightEdge (2024) reports that AI-generated summaries introduced through Google’s Search Generative Experience (SGE) have reduced click-through rates on organic listings by nearly 30\%. Simultaneously, Search Engine Land notes that over 60\% of queries now terminate without a user clicking any result—a phenomenon linked directly to the rise of AI-synthesized answers \cite{lodolce2024gartner}. This renders the notion of “ranking first” increasingly irrelevant when the user no longer sees or interacts with ranked links.

\subsection{Citation Behavior in Generative Systems}

Generative search engines such as Perplexity.ai, You.com, and ChatGPT Search combine retrieval mechanisms with LLMs to produce synthesized responses. Unlike traditional search, the visibility of source content is now determined by whether and how it is cited in the generated answer. In this context, citation is not merely a function of retrieval relevance, but of semantic alignment, modularity, and linguistic fluency.

Liu et al. \cite{liu2023verifiability} highlight that LLMs prioritize citations that are clear, factually self-contained, and coherent with the user’s query. However, their study also notes that current LLMs lack consistent grounding, leading to low citation recall and misattributions. To address this, Menick et al. \cite{menick2022quotes} propose reinforcement learning strategies to train models to support generated claims with verifiable quotes.

\subsection{The GEO Framework}

Aggarwal et al. \cite{aggarwal2024geo} provide the most comprehensive framework to date by defining visibility within generative engines as a multidimensional function. Their benchmark, GEO-Bench, consists of 10{,}000 queries across 25 domains, and introduces both objective (e.g., position-adjusted word count) and subjective (e.g., relevance, diversity, follow-up probability) impression metrics. They show that citations appearing earlier in a response are weighted more heavily by users, consistent with click decay models observed in earlier SEO research.

Their results indicate that GEO strategies such as adding statistics, improving fluency, and citing authoritative sources can improve citation metrics by up to 40\%—even for lower-ranked sources that would not typically benefit from traditional SEO ranking mechanisms.

\subsection{Gap in Implementation Research}

Despite recent progress, most GEO-related studies stop short of providing implementable methodologies. While Aggarwal et al. introduce benchmark evaluation metrics, they do not offer end-to-end implementation blueprints or field experiments in live production settings. Lüttgenau et al. \cite{luttgenau2025beyondseo} provide one such case by fine-tuning a BART model to optimize travel-related content, demonstrating a 30.96\% increase in position-adjusted citation word count over a baseline.

Our work complements these contributions by formalizing an actionable GEO framework grounded in a three-layer semantic model, and testing its effectiveness across multiple real-world deployments. Through both structured page engineering and longitudinal citation tracking, we aim to bridge the gap between generative theory and web practice.

\section{Theoretical Foundation: The Three-Layer Semantic Visibility Model}

3. Theoretical Foundation: The Three-Layer Semantic Visibility Model
The Generative Engine Optimization (GEO) framework is underpinned by a structured understanding of how large language models (LLMs) discover, interpret, and cite web content. Building on citation behavior observed in generative engines such as ChatGPT and Perplexity, we propose a three-layer semantic visibility model that captures the pathways through which content becomes quotable within generative responses. Each layer corresponds to a distinct mechanism in the LLM's information ingestion and synthesis pipeline, and maps directly to modifiable aspects of web content architecture.
\subsection{Layer 1: Semantic Anchoring}

3.1 Layer 1: Semantic Anchoring
Semantic Anchoring refers to the ability of content to be clearly classified and contextually grounded by the LLM during pre-retrieval and indexing phases. Empirical evidence from Aggarwal et al. (2024) suggests that generative engines favor sources with well-defined topical scope and explicit structural cues. To this end, content optimized for semantic anchoring must exhibit:

Descriptive and unambiguous titles, which explicitly state the topic or claim being made.

Introductory summary paragraphs, typically within the first 150–300 characters, that encapsulate the scope, key findings, and relevance of the page. These are best marked semantically using HTML tags such as <section class="summary"> or <p class="intro">.

Hierarchical heading structures (H1–H3), ensuring each section is semantically independent yet logically connected.

Semantic anchoring aligns with Liu et al. (2023)’s findings that LLMs perform better citation grounding when source content follows predictable structural patterns and contains salient topic signals.
\subsection{Layer 2: Context Triggering}

3.2 Layer 2: Context Triggering
Context Triggering addresses the retrievability of content across a broad spectrum of semantically equivalent or related queries. Unlike search engines that rely on keyword frequency and anchor text, LLMs rely on internal embeddings and semantic matchings. Thus, a page must include:

Synonymic and paraphrased phrasing of key ideas, to capture variant query intents (e.g., “GEO visibility” alongside “AI citation performance”).

Domain-specific terminology, including technical jargon and taxonomical language relevant to the field.

Multi-level complexity layering, where beginner-friendly explanations coexist with advanced analytical interpretations.

This design enables the content to surface regardless of user literacy level or phrasing strategy, a concept supported by Aggarwal et al. (2024) and operationalized in their GEO-Bench multi-domain query coverage model.
\subsection{Layer 3: Pragmatic Recomposition}

3.3 Layer 3: Pragmatic Recomposition
The final layer, Pragmatic Recomposition, ensures that content is modular and syntactically robust enough to be extracted, rephrased, or partially quoted by an LLM while preserving its semantic integrity. This layer is critical for maximizing inclusion in generative responses. Key features include:

Modular paragraphing, where each paragraph centers on one claim or concept and can be understood independently of surrounding text.

Q&A structures and FAQ blocks, which match the natural output format of many LLMs and are highly reusable in response generation. These should be marked with FAQPage schemas when possible.

List and step-wise formatting, useful for procedural and instructional content.

Standalone factual sentences, especially numerical results or definitions, which can be cited verbatim (e.g., “GEO citation uplift rate was 77.1% in post-optimization testing.”).

This approach is aligned with Lüttgenau et al. (2025)’s use of fine-tuned summarization models that were trained on (w, w′) pairs, where w′ embodied optimized variants of modular content annotated for fluency, authority, and citation suitability.
\subsection{Layer Synergy and Failure Modes}

3.4 Operational Interdependence
Although the three layers are analytically distinct, they are operationally interdependent. Content that is semantically anchored but not pragmatically modular may be retrieved but not cited. Conversely, highly modular content without semantic clarity may be cited out of context or not cited at all. We thus propose that effective GEO optimization requires simultaneous attention to all three layers, and should be evaluated using a multi-factor diagnostic framework as detailed in Section 5.

\section{Methodology: Technical Implementation of GEO}

To operationalize the three-layer semantic visibility model, we constructed a multi-stage GEO implementation pipeline using publicly available tools and lightweight deployment infrastructure. This section details the real-world execution process, emphasizing reproducibility, modularity, and platform neutrality.

\subsection{Content Generation and Structuring}

Initial content was drafted by dictating past experience and project descriptions using ChatGPT’s voice-to-text transcription. Supporting materials such as research notes and structured knowledge were integrated via ChatGPT’s linker and research tools.

Each content block was revised to:
\begin{itemize}
  \item Maintain modular paragraphing (3--5 sentences)
  \item Use informative, scoped H2/H3 headers
  \item Provide standalone blocks such as lists, FAQs, and short definitions
\end{itemize}

This directly supports Layer 1 and Layer 3 of the semantic visibility model.

\subsection{Deployment via Semantic Mesh}

Using Claude 3’s “Project Knowledge” export function, structured content was published as static HTML files. These were organized on GitHub Pages under a mesh architecture:
\begin{itemize}
  \item \textbf{Pillar nodes:} author profile, main topic overview
  \item \textbf{Cluster nodes:} grouped pages (e.g., SEO, LLM, case studies)
  \item \textbf{Mini nodes:} specific modules or tools
\end{itemize}

Each page interlinked upward toward its Pillar node, maintaining semantic cohesion and low crawl depth.

\subsection{Schema Integration}

Pages used Schema.org markup in JSON-LD format. Depending on content type, we added:
\begin{itemize}
  \item \texttt{Article} and \texttt{FAQPage} for explanatory and Q\&A blocks
  \item \texttt{Person} and \texttt{WebPage} for biography and overview pages
\end{itemize}

Markup was validated via Google’s Rich Results Test.

\subsection{Indexing and Monitoring}

Pages were submitted through Google Search Console. During the 7/6--7/15 observation period, indexation coverage was confirmed. Although average rank was low (position 10.7), citation occurred in generative responses before reaching top SERP.

\subsection{Summary of Toolchain}

\begin{center}
\begin{tabular}{|l|l|l|}
\hline
\textbf{Step} & \textbf{Tool} & \textbf{Output} \\\hline
Dictation & ChatGPT Voice & Modular paragraphs \\\hline
Knowledge Structuring & ChatGPT Linker & Topic clusters \\\hline
Export & Claude Project Knowledge & HTML pages \\\hline
Hosting & GitHub Pages & Semantic Mesh \\\hline
Schema Testing & Rich Results Test & JSON-LD validation \\\hline
Index Monitoring & Google Search Console & Coverage + rank data \\\hline
Evaluation & ChatGPT/Perplexity (incognito) & Citation rates \\\hline
\end{tabular}
\end{center}

\section{Evaluation Metrics} \label{sec:metrics}

To systematically assess the effectiveness of GEO interventions, we define five complementary evaluation metrics that measure different dimensions of content visibility within generative engines. These indicators are grounded in existing citation theories \cite{aggarwal2024geo, liu2023verifiability, menick2022quotes} and adapted for real-world observability and automation.

\subsection{AIO Semantic Focus Score}

\textbf{Definition:} The share of a document’s sentences that exhibit high semantic alignment with a declared topic or entity of focus.

Let $S$ be the total number of sentences and $S_t \subset S$ be those that contain or reinforce the primary topic entity $T$, as annotated via Named Entity Recognition (NER) or manual labeling.

\[
\text{AIO\_Semantic\_Focus}(T) = \frac{|S_t|}{|S|}
\]

\textbf{Threshold:} $\geq 0.75$

\textbf{Usage:} Diagnoses clarity of anchoring (Layer 1). A low score implies excessive topic drift.

\subsection{Citation Potential Score}

\textbf{Definition:} A composite metric scoring a paragraph or document on four citation-enhancing dimensions identified in GEO-Bench: factual density, authoritativeness, utility, and presentation clarity.

We adapt a simplified linear aggregation based on normalized human rater scores or GPT-assisted rubric scoring (following G-Eval templates).

\[
\text{Citation\_Potential} = \frac{1}{4} \sum_{i=1}^{4} \text{score}_i
\]

Each sub-score includes:
\begin{itemize}
  \item \textbf{Factuality}: Presence of numerical or time-bound facts
  \item \textbf{Authority}: Institutional tone or cited sources
  \item \textbf{Utility}: Actionable insights or definitions
  \item \textbf{Clarity}: Syntactic fluency and paragraph modularity
\end{itemize}

\textbf{Threshold:} $\geq 0.70$

\subsection{Structural Readiness Score}

\textbf{Definition:} A schema-based binary score evaluating the extent to which a page uses structured markup conforming to Schema.org types relevant for generative understanding.

\[
\text{Structural\_Readiness} = \frac{\text{Valid schema types used}}{\text{Expected schema types}}
\]

\textbf{Schema types considered:} \texttt{Article}, \texttt{FAQPage}, \texttt{Person}, \texttt{WebPage}

\textbf{Threshold:} $\geq 0.80$

\subsection{Modular Extractability Score}

\textbf{Definition:} The proportion of a document that is easily separable into extractable, stand-alone units such as questions, steps, bullet points, or key takeaways.

\[
\text{Modular\_Extractability} = \frac{\text{\# modular units}}{\text{\# total blocks (paragraphs + lists)}}
\]

\textbf{Modular units include:}
\begin{itemize}
  \item Stand-alone 3--5 sentence paragraphs
  \item Numbered or bulleted lists
  \item FAQ blocks
  \item Inline definitions or quote blocks
\end{itemize}

\textbf{Threshold:} $\geq 0.65$

\subsection{Multi-Modal Adaptability Score}

\textbf{Definition:} Measures whether the content contains elements that enable multimodal recomposition by LLMs—such as figure descriptions, audio summary outlines, or tabular data.

\[
\text{Multi\_Modal\_Adaptability} = \frac{\text{Alternate format sections}}{\text{Total sections}}
\]

\textbf{Detected features include:}
\begin{itemize}
  \item \texttt{<figure>} or \texttt{<table>} elements with captions
  \item Podcast/video outlines (bulleted)
  \item Data tables with headers
\end{itemize}

\textbf{Threshold:} $\geq 0.60$

These metrics align with and expand on the visibility functions and impression metrics introduced in GEO-Bench \cite{aggarwal2024geo}. They support reproducible benchmarking across domains and platforms.

\section{Case Study: GEO Implementation on “SOYA 課程評價” Content}

To validate the practical efficacy of Generative Engine Optimization (GEO), we conducted a three-week structured implementation on a public-facing web page discussing the “SOYA 課程評價” (SOYA course reviews). The process was executed in July 2024 and documented step-by-step at \url{https://www.massenlighten.com/post/gpt-%E5%BC%95%E7%94%A8%E5%AF%A6%E4%BD%9C%E5%85%A8%E7%B4%80%E9%8C%84%EF%BC%9A%E4%B8%89%E9%80%B1%E5%BE%9E%E9%9B%9C%E5%87%BA%E7%99%BC%EF%BC%8C%E6%89%93%E9%80%A0-ai-%E8%83%BD%E7%9C%8B%E8%A6%8B%E7%9A%84%E5%85%A7%E5%AE%B9}.

\subsection{Pre-Optimization Baseline}

Prior to optimization, we queried “SOYA 課程評價” in both \textbf{ChatGPT Search} and \textbf{Perplexity AI} in incognito and non-authenticated visitor mode.

\begin{itemize}
  \item \textbf{ChatGPT:} 0 of 8--10 answer segments cited massenlighten.com.
  \item \textbf{Perplexity:} 0 of 6 response segments cited our content.
\end{itemize}

Although the SOYA page ranked highly on Google (position 1), it was not referenced by generative engines—demonstrating that SEO visibility did not translate into LLM citation.

\subsection{Three-Week GEO Optimization}

We applied structural interventions mapped to each GEO semantic layer:

\begin{itemize}
  \item \textbf{Week 1 (Semantic Anchoring):} Rewrote title and introduction; added summary section using \texttt{<section class="summary">}.
  \item \textbf{Week 2 (Context Triggering):} Inserted paraphrases (e.g., “課程評價” vs. “學員回饋”), related terminology, and domain synonyms.
  \item \textbf{Week 3 (Pragmatic Recomposition):} Broke paragraphs into 3--5 sentence units, added FAQ blocks and bolded definitions.
\end{itemize}

\subsection{Post-GEO Evaluation Results}

We re-ran the same queries under identical visitor-mode conditions.

\begin{table}[h]
\centering
\begin{tabular}{|l|c|c|}
\hline
\textbf{Engine} & \textbf{Answer Segments} & \textbf{Citing Our Page} \\\hline
ChatGPT (Run 1) & 9 & 7 \\\hline
ChatGPT (Run 2) & 8 & 7 \\\hline
Perplexity (Run 1) & 6 & 5 \\\hline
Perplexity (Run 2) & 6 & 5 \\\hline
\end{tabular}
\caption{Citation results across 4 zero-login visitor sessions (July 2024)}
\end{table}

Average citation ratio across both engines and both runs was \textbf{77.1\%}.

\subsection{Interpretation}

The experiment confirms that:

\begin{itemize}
  \item \textbf{Semantic modularity} was key to citation inclusion;
  \item \textbf{Citation occurred despite no change in Google SEO rank or backlinks};
  \item \textbf{LLMs reused both rewritten paragraphs and FAQ blocks}, showing effective recomposition.
\end{itemize}

These findings match GEO-Bench and BART fine-tuning results reported by Aggarwal et al. \cite{aggarwal2024geo} and L\"uttgenau et al. \cite{luttgenau2025beyondseo}, where modular content with semantic clarity increased AI citation performance even without traditional SEO indicators.

\section{Control Experiment: Name-Based GEO Visibility Test}

To isolate the effects of Generative Engine Optimization (GEO) from traditional Search Engine Optimization (SEO), we conducted a single-subject case study using the author’s name, “Tianyou Liao” (original in Chinese) as the query term. Unlike product or course-related keywords, personal names are less susceptible to keyword targeting or backlink manipulation, making them a suitable context for testing semantic-layer interventions under low-SEO influence.

\subsection{Motivation and Hypothesis}

Prior to this experiment, searching “Tianyou Liao” on ChatGPT and Perplexity produced zero references to the author. Instead, both engines returned results about a different individual---a well-known student with extensive news coverage. This high-ambiguity, low-authority setting provided an ideal stress test for GEO strategies.

We hypothesized that even in the presence of a dominant competing entity and low page rank, semantic anchoring and modular structuring alone could elevate visibility in generative responses.

\subsection{Implementation Pipeline (7/6--7/15)}

The author executed a full-stack GEO deployment using the following sequence:

\begin{enumerate}
  \item Used ChatGPT voice transcription to capture biography and work history.
  \item Structured personal notes and knowledge via ChatGPT linker and research tools.
  \item Exported content as static HTML using Claude 3’s Project Knowledge tool.
  \item Published content on GitHub Pages with semantic interlinking (Pillar/Cluster/Mini nodes).
  \item Embedded \texttt{Person}, \texttt{FAQPage}, and \texttt{Article} schema using JSON-LD.
  \item Submitted site to Google Search Console on July 6; monitored indexation and ranking.
\end{enumerate}

No SEO enhancements (e.g., backlinks, keyword stuffing, metadata) were used. This isolated the effect of semantic and structural design.

\subsection{Evaluation Results (7/15)}

\begin{itemize}
  \item \textbf{Google Search Console}: Indexed with average position = 10.7.
  \item \textbf{Google organic search:} Author site appeared on page 3 (approx. rank 25--30).
  \item \textbf{ChatGPT:} Returned 9 content segments, 5 of which cited the author’s pages.
  \item \textbf{Perplexity:} Returned 5 content blocks, 3 of which cited the author.
\end{itemize}

Notably, ChatGPT’s response explicitly included:

\textit{“There are two notable individuals named Tianyou Liao”}

---referring to both the widely known student and the author, indicating successful semantic disambiguation via Schema and structured content.

\subsection{Interpretation}

This follow-up experiment confirms that:

\begin{itemize}
  \item GEO-structured content achieved over \textbf{55\% citation} without SEO ranking.
  \item Semantic anchoring using \texttt{Person} schema enabled LLM disambiguation.
  \item Modular paragraphing and clear topical labels enabled partial citation even in the presence of a dominant name collision.
\end{itemize}

The results support findings from GEO-Bench \cite{aggarwal2024geo}, indicating that citation in generative systems is governed primarily by semantic extractability and structural clarity rather than by popularity, backlink profile, or keyword dominance.

\section{Discussion}

The preceding experiments reveal a consistent and reproducible pattern: content optimized using the GEO framework demonstrates substantial improvements in citation frequency within generative engine outputs---regardless of its SEO status, backlink presence, or domain authority. This section synthesizes the theoretical implications, operational considerations, and real-world constraints identified across both case studies.

\subsection{From Ranking to Referencing}

The SOYA course trial demonstrated that even when a page is already ranked highly in traditional search (e.g., first result on Google), it may remain completely invisible to generative engines like ChatGPT or Perplexity if it lacks semantic modularity and extractability. After applying structured Answer Layer rewriting, schema markup, and paragraph modularization, the same content was cited in over 77\% of generated segments---a finding that mirrors the citation uplift seen in GEO-Bench \cite{aggarwal2024geo}.

In contrast, the “Tianyou Liao” name-based trial involved a page that ranked poorly in SEO (GSC average position 10.7) and shared its target query with a dominant, unrelated entity. Even so, the GEO-structured content achieved a 60\%+ citation rate in both generative engines---suggesting that well-engineered semantic structure can overcome even \textbf{disambiguation bias}.

Together, these findings support the core GEO hypothesis: \textbf{LLM visibility is governed more by extractability, coherence, and topical clarity than by link-based authority signals}.

\subsection{Layer Effectiveness and Interaction}

The experiments further validate the functional independence and interaction of the three GEO semantic layers:

\begin{center}
\begin{tabular}{|l|l|l|}
\hline
\textbf{Layer} & \textbf{Contribution Evidence} & \textbf{Observed Effects} \\\hline
Semantic Anchoring & Clear titles and summaries & Higher classification accuracy \\\hline
Context Triggering & Synonymic phrasing, taxonomy & Broader query matching \\\hline
Pragmatic Recomposition & FAQs, modular paragraphs & Mid-body citation precision \\\hline
\end{tabular}
\end{center}

Notably, ChatGPT cited not only the introductory paragraphs, but also embedded FAQ answers and sentence-level definitions---reinforcing the role of Layer 3 in enabling recomposition. This behavior is consistent with Lüttgenau et al.’s BART fine-tuning results \cite{luttgenau2025beyondseo}.

\subsection{Visibility \texorpdfstring{$\neq$}{≠} Authority}


One counterintuitive outcome is the citation of low-authority domains (e.g., GitHub Pages with no backlinks) over highly ranked official sites. This challenges a long-standing SEO assumption: that PageRank correlates with trust or visibility.

In GEO, trust emerges from \textbf{information packaging}, not \textbf{linkage popularity}. This suggests that smaller or emerging creators can compete on equal footing if they invest in semantic clarity and modular formatting.

\subsection{Constraints and Limitations}

\begin{itemize}
  \item \textbf{LLM output variability:} Generative results may vary slightly depending on prompt, model version, or context.
  \item \textbf{Crawling uncertainty:} It remains impossible to directly verify when an LLM ingested a newly published site.
  \item \textbf{Black-box citation logic:} LLMs may cite based on latent embeddings or internal heuristics not fully transparent to users.
\end{itemize}

These limitations are consistent with the “partially observable” nature of GEO described by Aggarwal et al. \cite{aggarwal2024geo}, and reinforce the need for multi-session testing and layered metric evaluation.

\subsection{Validation of GEO's Three-Layer Model}

Across all case studies and experiments conducted, our results provide strong empirical support for the functional effectiveness of GEO’s proposed three-layer semantic visibility framework. Specifically, we observe:

\begin{itemize}
  \item \textbf{Layer 1: Semantic Anchoring} was consistently validated across domains. Clear topical headings, descriptive summaries, and introductory framing led to robust inclusion in generative outputs, even when other structural formatting was absent (Sections 6.1, 7.4).
  
  \item \textbf{Layer 3: Pragmatic Recomposition} proved to be the most reliable citation trigger, particularly in FAQ-style queries. Modular paragraph design, list formatting, and structured Q\&A blocks significantly enhanced citation likelihood (Sections 6.1–6.3).
  
  \item \textbf{Layer 2: Context Triggering}, initially less observable in single-domain settings, was subsequently validated in Section 6.5 via a dedicated cross-domain experiment. We demonstrated that paraphrastic phrasing, synonymous terminology, and taxonomic generalizations substantially improved recall across semantically varied queries.
\end{itemize}

Together, these findings confirm that the three layers, while distinct in mechanism, operate synergistically to influence LLM citation behavior. Effective content design for generative engine visibility therefore requires simultaneous attention to anchoring clarity, semantic breadth, and modular composition.


\section{Conclusion}

This study presents and empirically validates a structured framework for Generative Engine Optimization (GEO), introducing a three-layer model of semantic visibility: Semantic Anchoring (Layer 1), Context Triggering (Layer 2), and Pragmatic Recomposition (Layer 3). Through multi-stage interventions and simulations across three real-world scenarios—including a course review site, a name disambiguation case, and a large-scale cross-domain semantic variation test—we observed measurable gains in citation inclusion within AI-generated outputs. Our approach demonstrates that citation is not merely a byproduct of search engine ranking, but a function of how content is semantically structured and expressed.


\section{Conclusion and Future Work}

This study formalizes and operationalizes Generative Engine Optimization (GEO) as a reproducible content strategy for visibility in LLM-powered search. Through two real-world deployments---SOYA course content and a name-based disambiguation scenario---we demonstrated measurable citation improvements using only structural and semantic techniques, independent of SEO rank or backlinks.

\textbf{Key contributions:}
\begin{itemize}
  \item A three-layer semantic visibility model integrating anchoring, triggering, and recomposition.
  \item Implementation guidelines using ChatGPT, Claude, GitHub Pages, and Schema.org.
  \item Five evaluation metrics for measuring semantic and structural citation readiness.
  \item Empirical results confirming citation uplift of up to 77.1\% in ChatGPT and Perplexity.
\end{itemize}

These findings affirm that GEO enables visibility parity even in low-authority scenarios and provide a practical foundation for LLM-era content design.

\subsection{Future Directions}

Several avenues merit further investigation:

\begin{itemize}
  \item \textbf{Citation Monitoring Tools:} Browser extensions or LLM-integrated dashboards that detect if and how a URL is cited across sessions.
  \item \textbf{Automatic GEO Scoring:} Tools that use LLMs (e.g., GPT-4, Claude 3) to compute scores like AIO, CPS, and SRS directly from raw HTML inputs.
  \item \textbf{Cross-Engine Testing:} Replication of these experiments across other generative platforms (e.g., You.com, Brave AI, Meta AI) to test framework generalizability.
  \item \textbf{RL-based Optimization:} Fine-tuning small models on citation-based reward functions to iteratively restructure pages for maximum generative visibility \cite{luttgenau2025beyondseo}.
\end{itemize}

As generative engines grow in influence, understanding and influencing how content is ingested and cited becomes a critical capability. GEO offers both a theoretical framework and operational toolkit for this new paradigm of search and citation.

\section*{Postscript: Reverse Experiment – SEO Without GEO}

To further validate the necessity of semantic structuring in GEO, we conducted a reverse experiment using a low-competition English name: \textbf{Tsai Hao Jui}.

An article on AI citation optimization was authored under this name, but deliberately \emph{excluded} all structural techniques introduced in the GEO framework—no modular paragraphs, no Schema.org markup, and no semantic mesh linking.

Due to the rarity of the name, the page easily achieved the top SEO position for the query ``Tsai Hao Jui''. However, the citation visibility results revealed a clear gap:

\begin{itemize}
  \item \textbf{Google:} The article ranked first for ``Tsai Hao Jui''.
  \item \textbf{ChatGPT:} Returned no information about the authored content.
  \item \textbf{Perplexity:} Cited only a LinkedIn profile; the article content was not referenced.
\end{itemize}

These results confirm that \textbf{SEO ranking alone is insufficient for generative engine citation}. Without semantic extractability and modular design, even high-ranking content remains invisible to LLM-based systems.

Screenshot evidence for this experiment is included as:

\begin{itemize}
  \item \texttt{figures/case3-tsai-chatgpt.png}
  \item \texttt{figures/case3-tsai-google.png}
  \item \texttt{figures/case3-tsai-perplexity.png}
\end{itemize}

This reinforces the core hypothesis of this paper: visibility in generative systems is governed not by surface ranking, but by structured semantic readiness.


\bibliographystyle{plainnat}
\bibliography{references}

\end{document}

